
\section{Introduction}

DruL stands for ``Drumming Language''.  It is a programming language designed for composing drum music.  It is common these days for drum beat composers to create drum parts using computer software (e.g. FL Studio).  Creating drums parts with these programs often involes of lot of tedious ``pointing and clicking'' (especially when making longer drum parts).  DruL was designed to give the composer the ability to automate much of this tedium.  There already exist other more general-purpose music programming languages (e.g. ChucK, SuperCollider, Nyquist, Haskore).  These languages are complicated by note pitches, durations, and audio effects.  DruL is unconcerned with these things and focuses soley on allowing the drum composer to define and manipulate beat patterns.
