% lessons learned, one section per team member

\chapter{Lessons Learned}

\section{Introduction}
In this chapter each team member tells about some lessons he learned from
the project, and what he would do differently if we had start all over
again.

\section{Rob (team leader)}

\section{Ben}

\section{Thierry}
One part of the code I especially worked on is the test suite, but I still
was surprised to see how important it turned out to be. In a new project,
I would either build a more powerful testing program, or spend more
time to find an appropriate package online. For instance, our current
testing program does not have the ability to test an output file instead
of the standard output. It would have become a problem if our language was
designed for file operations.

Another lesson learned is the importance of helper functions designed early.
At one point, every one of us had design is own method to lookup into the
environment, and obviously we multiplied the number of bugs. For some
functions, it is so obvious that they were going to be needed that we should
have spent the time, as a team, to define them. Their documentation is
also an important aspect when you work in a team of more than two programmers.

Following that idea, we probably did not use enough the ``issue tracking''
on Google code, the platform we used to host our project. Emails does not
work as well...

\section{Waseem}

Most important lesson in while coding in OCaml is to modularize the code.
Those match with clauses keep getting messier and also there is a lot of code 
repeatition while implementing similar functions or methods on the same language 
type object, e.g., pattern, clip, etc. Therefore, it is always good to have the 
helper functions, that can be used later on, in the code. This was my first group
project of this level and believe it or not, my first time using version control:); 
Really makes your life easier. Of course, having those lexer and parser tools do 
most of the work for you is vry helpful. OCaml in itself is a rather powerful language. 
Syntax tends to get 'messy', however, its power is well to be noted. The code tends 
to be compact, especially when you factor out code that is repeated.

Working in pairs is definitely more helpful than working on one thing alone. In the former
case you less likely tend to get stuck at a point, as compared to the later case. 

