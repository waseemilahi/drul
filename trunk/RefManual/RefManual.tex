\documentclass[11pt,twoside]{article}
 
\usepackage{hyperref}
\usepackage{graphicx}
\usepackage{amsmath}

\pagestyle{plain}

\setlength{\voffset}{0pt}
\setlength{\marginparwidth}{0pt}
\setlength{\oddsidemargin}{30pt}
\setlength{\evensidemargin}{30pt}
\setlength{\textwidth}{400pt}
\setlength{\parindent}{0pt}
\setlength{\parskip}{10pt plus 2pt minus 1pt}

\title{\textbf{DruL} Reference Manual\\
\vspace{1cm}
COMS W4115: Programming Language and Translators}
\author{Team Leader: Rob Stewart (rs2660) \and Thierry Bertin-Mahieux (tb2332) \and Benjamin Warfield (bbw2108) \and Waseem Ilahi (wki2001)}

\begin{document}
\maketitle
\begin{center}
\end{center}

\vspace{3cm}

\begin{figure}[h]
\begin{center}
\includegraphics[width=.2\columnwidth]{Water_Drop.pdf}
\end{center}
\end{figure}


\newpage


\section{Introduction}
DruL is a programming language designed for composing drum music.  Unlike other more general-purpose music programming languages (ChucK, SuperCollider, Nyquist, Haskore), DruL's focus is on defining and manipulating beat patterns and is unconcerned with pitches, sound durations, or audio effects.

DruL is mainly an imperative programming language, however it borrows ideas (map and filter) from the functional paradigm.  In additions to integers, DruL's main datatypes are pattern and clip. Instruments are defined as constants.  See below for further details.

DruL programs do not contain any loops or user-defined functions.  All pattern 
and clip creation and manipulation is done using the map construct described below.

\section{Language Manual}

\subsection{Lexical Conventions}

\subsubsection{Comments}
Comments start with // and everything after them on that line
is considered a comment and is ignored. New line marks the end
of a comment.
\subsubsection{Whitespace}
Spaces, tab, end of line, and return are all considered the same 
and their only used is to seperate the tokens.

\subsubsection{Characters}
DruL uses the ASCII character set.

\subsubsection{Identifiers}
An identifier may be defined using any uppercase or lowercase character
or an underscore, followed by any set of uppercase or lowercase character,
underscore, and digits ($0$ through $9$). The maximum length of an identifier
is $64$.

\subsubsection{Keywords}


\begin{table}[htb]
\begin{center} 
\begin{tabular}{llll} 
  NULL &    rand &    clip &        mapper \\
  if &      pattern & instrument &  print \\
  elseif &  concat &  length &      output \\
  else &    slice &   map &         return
\end{tabular}
%\caption{\small{.}}
\label{tab:keywords}
\end{center} 
\end{table}


\subsection{Types}

There are $3$ types in DruL: \textbf{integer}s, \textbf{pattern}s,
and \textbf{clip}s.  In addition, string constants may be used in DruL source code, but there is no variable type to which they can be directly assigned.
Values in DruL are strongly typed, but the type of a variable is determined dynamically.

\subsubsection{integer}
All integers are base 10, and may optionally be preceded by a sign ({\tt \textbf + -}).
Any sequence of digits ($0$ through $9$) is valid.  Leading $0$s are ignored, so a sequence such as $0000123$  is interpreted as $123$.  Integers are mutable.

\subsubsection{pattern}
A pattern is essentially an object that holds binary, discrete, time-series data.  At each discrete-time step, which will henceforth refer to as a beat, there is either a note or a rest.  For the non-musically inclined, a note represents sound produced by the striking of a drum (or similar instrument) and a rest represents the absence of any such sound.  Patterns are immutable.  When a pattern is manipulated, the target pattern remains intact and a new copy is created.  The length of a patten can be any non-negative integer.

\subsubsection{clip}

An instrument is one of a pre-defined set of sounds (e.g. drum notes) that can occupy a single beat.
A clip is a mapping of patterns to instruments. Clips are processed in sequence as the program runs to produce output which may be audio, sheet-music notation, or a MIDI file.  Clips are immutable.

\subsubsection{string}

A string constant begins with an ASCII double-quote character, continues with an arbitrary sequence of ASCII characters other than $\\$ and $''$, and concludes with another $''$ character.  If a $\\$ or $''$ character is desired, it can be escaped using the $\\$ character.

\subsection{Statements}

In the most common case, a statement consists of a single expression followed by a semicolon (``;'').  Importantly, unlike many languages with similar syntax, DruL does \emph{not} consider a block to be equivalent to a statement.  Instead, statements in DruL take one of the four forms below.



\subsubsection{Expression Statements}

The basic form of statement, as in most C-like languages, is the expression statement: \emph{expression-statement}: \emph{expression}\textbf{;}

The precedence table for operators in DruL is given here:

\begin{tabular}{ l |l| l}
\hline\hline
Operators & Notes & Associativity \\
\hline $ . $ & Method call & left to right \\
$-$  $!$ &Unary minus and logical negation & right to left \\
$*$ $/$ $\%$ & Standard C meanings & left to right \\
$+$ $-$ & Addition/subtraction & left to right \\
$<$ $<=$ $>$ $>=$ &  &  \\
$!=$ $==$ &  & \\
$\&\&$ &  & \\
$||$ &  & \\
\end{tabular}

The subsections that follow use the model of the C Language Reference Manual to indicate the various types of expression.  As in that example. the highest-precedence forms of expression are listed first.

\paragraph{Primary Expressions}

A primary expression consists of a constant (integer or string), an identifier, or a parenthesized expression.

\paragraph{Function and Method calls}
Right to left.

\paragraph{Unary operations}
Arithmetic and logical negation.

\paragraph{Standard arithmetic operations}

\paragraph{Comparison operations}

(relation first, then equality)

\paragraph{Logical combination operations}

AND precedes OR.

\subsubsection{Assignment Statements}

\[ assignment-statement: identifier \mathbf{=} expression-statement \]

Assignment in DruL is not a simple operator to be placed in the middle of an expression.  Rather, it is a separate type of statement, which may appear anywhere another statement may appear.  

\emph{assignment-statement}: \emph{identifier} \textbf{=}  \emph{expression-statement}

\subsubsection{Selection Statements}

Selection statements in DruL take the following form: the string \textbf{if}, followed by an expression that returns a boolean result, enclosed in parentheses, followed by an open-brace (``\{''), one or more statements, and a close-brace (``\}'').  This may optionally be followed by one or more \textbf{elseif}s, which are also followed by parentheses and a block of statements, and one \textbf{else}, which omits the test expression but is also followed by a block of statements.

\begin{verbatim}
if (expression) { statement-list } 
[ elseif (expression) { statement-list } ]*
else { statement-list }
\end{verbatim}

\subsubsection{Mapper Definition Statements}

A mapper definition consists of the word \textbf{Mapper}, followed by an identifier, followed by a parenthesized list of space-separated identifiers, followed by a block.

\begin{verbatim}
Mapper name (arg1 arg2  arg3)
{
    statement-list
};
\end{verbatim}

\subsubsection{Blocks}
DruL has a limited block structure: only in the context of an \textbf{if}/\textbf{elseif}/\textbf{else} sequence or a Mapper Definition statement is a new block needed or allowed.  In these cases, curly braces (``\{\}'') are used to delimit the statement-sequence that falls within the block, and they must contain one or more statements.

Mapper definitions define a new scope, from within which externally defined variables are not visible; \textbf{if} blocks do not define a new scope, so all variables used within them are visible to the enclosing  block, and vise-versa.

\subsection{Patterns and pattern operations}

\subsubsection{Patterns}
A pattern is a set of beats, each beat can be a note or a rest.
To declare a pattern, DruL uses `0' for rests and `1' for note.
A pattern can be created in the following way:
\begin{verbatim}
  p1 = pattern("101010");
\end{verbatim}
which represents the sequence {note, rest, note, rest, note, rest}.
Its length is $6$.

There are simple operations of patterns included in DruL. We
can repeat a pattern:
\begin{verbatim}
  p2 = pattern("100");
  p3 = p2.repeat(3);
  p4 = pattern("100").repeat(3);
\end{verbatim}
p3 is "100100100" and so is p4. Two patterns can be concatenated:
\begin{verbatim}
  p5 = pattern("111");
  p6 = pattern("000");
  p7 = concat(p5 p6);
\end{verbatim}
p7 is "111000". \textbf{concat} can takes any positive number of
patters as argument, patterns are concatenated from left to right.
Finally, you can have an empty pattern of length 0:
\begin{verbatim}
  p8 = pattern("");
\end{verbatim}


\subsubsection{Map}
A map is used to perform an operation iteratively on a set of patterns.
Patterns are iterated over from left to right. The output of a map is
a pattern. For example:
\begin{verbatim}
  p9 = pattern("101");
  p10 = map(p9) {if (p9.note) {return pattern("11")'} else {return  pattern("0");}}
\end{verbatim}
p10 is "11011". DruL uses \textbf{p.note} or \textbf{p.rest} to check whether
the current beat is a note or a rest, 'p' being the name of the input pattern.


\subsubsection{Mapper}
A mapper is the body of a map. It is used when a same operation has to be
used multiple times, over different sequences.
For example, the previous example could have been written in the following
way:
\begin{verbatim}
  mapper mymapper (p)
  {
      if (p.note) { return pattern("11"); }
      else        { return pattern("0");  }
  }
  p10 = map(p9) mymapper;
\end{verbatim}
Mapper has to be declared before it is used in the file.

\subsection{Clips}

\subsubsection{Instruments}

Before we define any clips, we must tell the compiler what instruments they will use.
This can only be done once per program, and uses the instruments function.  This function is unique in that it can take a variable number of arguments.  Each argument is the name of an instrument to be defined.  In the example below, four instruments are defined:

\begin{verbatim}
instruments(hihat bassdrum crash snare);
\end{verbatim}

Instruments must be defined at the beginning of a DruL program, before any clips have been defined.

\subsubsection{Clips}

A clip represents a collection of patterns to be played in parallel, where each pattern is played on a single instrument.

Once the instruments are defined, we can create a clip from our existing patterns, using an
associative-array notation:

\begin{verbatim}
clip1 = clip
(
    bassdrum = downbeats
    hihat    = alternate_beats
);
\end{verbatim}

The same result can be achieved by simply listing the patterns for each instrument in the order they are defined in the \textit{instruments} declaration:
\begin{verbatim}
clip2 = clip
(
    alternate_beats
    downbeats
    // remaining instruments have an empty beat-pattern
);

\end{verbatim}




\subsection{Outputs}
DruL has two kinds of outputs, the first one display a straightforward
representation of some data, the second one transform a clip into a
more complex representation, like MIDI.

\subsubsection{Standard output}
The \textbf{print} statement displays any type to the standard
output, including strings. For example:
\begin{verbatim}
  print "DruL";
  print pattern("01");
\end{verbatim}
The representation of a string is the string itself. The representation
of a pattern is the string that would have been used to initialized
the pattern. For example, if we have a pattern
\begin{verbatim}
  p = pattern("01").repeat(2);
\end{verbatim}
it's representation is \textbf{0101}.

\subsubsection{Text}
Similar as \textbf{print}, DruL can output the representation of any type 
to a file. The command is:
\begin{verbatim}
  output.file("myfile.txt","DruL");
  output.file("myfile.txt",pattern("01");
\end{verbatim}

\subsubsection{MIDI}
The function \textbf{output.midi(clip)} outputs a clip as a MIDI file,
default filename is "output.midi". 
The function \textbf{output.midi(clip,"my.file")} outputs a clip as a MIDI
file with filename "my.file". The transformation from clip to MIDI may rely
on external libraries as MIDGE\footnote{\url{http://www.undef.org.uk/code/midge/}}. There is no guarantee on which of the three existing MIDI formats is used.


\section{Examples}
In this section we give examples of what DruL code will look like, in the form
of a tutorial.

\subsection{Integers}

Integer values can be assigned to variables and used in expresisons as follows:

\begin{verbatim}
a = 3;
b = a + 2;
c = b * 12;
\end{verbatim}

\subsection{Pattern}
Patterns are the data type the programmer will likely spend most of their time dealing with.  For convenience, the programmer can supply a string constant made up of 1s and 0s, which will be translated into a pattern: if the character is a 1, there is a note on the corresponding beat; if 0, a rest.
\begin{verbatim}
p1 = pattern("101010");
\end{verbatim}
Patterns can be concatenated to form new patterns:
\begin{verbatim}
pcat = concat(p1 pattern("111000") pattern("1"));
\end{verbatim}
\textit{pcat} will be equal to 1010101110001.

There is also a shortcut to concatenate the same pattern many times:
\begin{verbatim}
pcat2 = concat(p1 p1 p1);
pcat3 = pattern("101010").repeat(3);
pcat4 = p1.repeat(3);
\end{verbatim}
\textit{pcat2}, \textit{pcat3}, and \textit{pcat4} are all equivalent.

\subsection{Map}
Of course, we will not hardcode every pattern we want to create. We use
map to create meaningful new patterns from existing ones:
\begin{verbatim}
p2 = map (p1)
{
    if (p1.note) { pattern("11"); }
    else         { pattern("0");  }
};
\end{verbatim}
This will create the following pattern: 110110110. The goal of a map
is to easily iterate over a pattern. \textit{p1.note} returns
\textit{true} if there is a note on the current beat, \textit{false} otherwise.
If you call map on multiple patterns that are not of the same length,
the shorter patterns will be padded with \textit{NULL} beats.\\

\subsection{Mapper}
For ease of use, you can define a \textit{mapper} that contains the behaviour
used by \textit{map}. We create \textit{p3}, which is the same as
\textit{p2}:
\begin{verbatim}
mapper myMapper (p1)
{
    if (p1.note) { return pattern("11"); }
    else         { return pattern("0");  }
}

p3 = map (p1) myMapper;
\end{verbatim}
\textit{mapper} will be very important when building a standard library
for the language.

\subsection{More complex examples}
Now that we have a proper syntax, let's get to more complicated examples.
We introduce $2$ new features that can be used inside a \textit{map}:
\textit{prev} and \textit{next}. They give you access to earlier
and later beats in a pattern, using the syntax \textit{p.prev(n)} and
\textit{p.next(n)}. 
%Also, for a pattern \textit{p}, \textit{p.rest} is
%true if and only if we did not reach the end of this pattern.

\textbf{reduction}: accelerate by cutting one beat out of two
\begin{verbatim}
downbeats = pattern("1000100010001000");
alternate_beats = pattern("10").repeat(8);
downbeat_diminution = map(downbeats alternate_beats)
{
    if     (alternate_beats.rest) { return pattern("");  } // pattern of length 0
    elseif (downbeats.note)       { return pattern("1"); }
    else                          { return pattern("0"); }
}
\end{verbatim}
output is: 10101010.

\textbf{improved reduction}: putting a rest (0) only if the $2$ original beats were rest
\begin{verbatim}
// this will map "1001100110011001" to "11111111", rather than "10101010"
one_and_four = pattern("1001100110011001");
alternate_beats = pattern("10").repeat(8);
improved_diminution = map(one_and_four alternate_beats)
{
    if     (alternate_beats.rest)      { return pattern("");  } // still required
    elseif (one_and_four.note)         { return pattern("1"); }
    elseif (one_and_four.next(1).note) { return pattern("1"); }
    else                               { return pattern("0"); }
};
\end{verbatim}

\end{document}
