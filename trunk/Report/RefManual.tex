%!TEX root =  Report.tex
\chapter{Language Reference Manual}
\section{Introduction}

DruL is mainly an imperative programming language, however it borrows ideas (map and filter) from the functional paradigm.  In addition to integers, DruL's main datatypes are pattern and clip. Instruments are defined as constants.

DruL programs do not contain any loops or user-defined functions.  All pattern
and clip creation and manipulation is done using the map construct described below.

\section{Language Manual}

\subsection{Lexical Conventions}

\subsubsection{Comments}

Comments in DruL start with the token ``//'' and continue until the end of the current line.
DruL has no multi-line comment syntax.

\subsubsection{Whitespace}

Space, tab, end of line, and return are all considered the same
and their only use is to seperate tokens.

\subsubsection{Characters}

DruL uses the ASCII character set.

\subsubsection{Identifiers}

An identifier consists of any uppercase or lowercase character
or an underscore, followed by any sequence of uppercase or lowercase characters,
underscores, and digits ($0$ through $9$). The maximum length of an identifier
is $64$ characters.

In addition, within the context of a mapper, special variables \$1 through \$$n$ (where $n$ is the number of patterns passed to the mapper) are defined as read-only aliases (see section \ref{MapSection} for more details on this feature).

All identifiers in a given scope, be they mapper names, variables, or built-in functions, belong to a single namespace.

\subsubsection{Keywords}

\begin{table}[htb]
\begin{center}
\begin{tabular}{llll}
  return & rand    & clip       & mapper \\
  if     & pattern & instrument & print  \\
  elseif & concat  & map        &    \\
  else   & true    & false      &
\end{tabular}
%\caption{\small{.}}
\label{tab:keywords}
\end{center}
\end{table}



\subsection{Types}

There are $3$ basic types in DruL: \textbf{integer}s, \textbf{pattern}s,
and \textbf{clip}s.  In addition, `string' constants may be used in DruL source code, but there is no variable type to which they can be directly assigned.
Likewise, boolean expressions exist, but cannot be assigned to variables.
Values in DruL are strongly typed, but the type of a variable is determined dynamically.

\subsubsection{integer}

All integers are base 10, and may optionally be preceded by a sign ({\tt \textbf + -}).
Any sequence of digits ($0$ through $9$) is valid.  Leading $0$s are ignored, so a sequence such as $0000123$  is interpreted as $123$.  Integers are mutable.

\textbf{rand} is a function that returns a non-negative number.  It either accepts a positive integer argument, in which case it returns a random number between 0 (inclusive) the argument (exclusive), or no argument in which case \textbf{rand} returns either 0 or 1.
\begin{verbatim}
r = rand();
s = rand(19);
\end{verbatim}
\subsubsection{pattern}

A pattern is essentially an object that holds binary, discrete, time-series data.  At each discrete-time step, which will henceforth be referred to as a beat, there is either a note or a rest.  For the non-musically inclined, a note represents sound produced by the striking of a drum (or similar instrument) and a rest represents the absence of any such sound.  Patterns are immutable.  When a pattern is manipulated, the target pattern remains intact and a new copy is created.  The length of a patten can be any non-negative integer.

\subsubsection{beat}

A beat is a lightweight object that cannot be created directly by the user: it exists only within a mapper (for more discussion of which, see section \ref{MapSection} below).  It gives direct access to information about a single beat of a \textbf{pattern} object (including the beats surrounding it).

\subsubsection{clip}

An instrument is one of a pre-defined set of sounds (e.g. drum notes) that can occupy a single beat.
A clip is a mapping of patterns to instruments. Clips are processed in sequence as the program runs to produce output which may be plain-text or a MIDI file.  Clips are immutable.

\subsubsection{string}

A string constant begins with an ASCII double-quote character, continues
with an arbitrary sequence of ASCII characters other than $\backslash$, $''$, and the ASCII newline character,
and concludes with another $''$ character.  If a $\backslash$ or $''$ character is
desired, it can be escaped using the $\backslash$ character.

\subsection{Statements}

In the most common case, a statement consists of a single expression followed by a semicolon (``;'').  Importantly, unlike many languages with similar syntax, DruL does \emph{not} consider a block to be equivalent to a statement.  Instead, statements in DruL take one of the six forms below.

\subsubsection{Expression Statements}

The basic form of statement, as in most C-like languages, is the expression statement: \emph{expression-statement}: \emph{expression}\textbf{;}

The precedence table for operators in DruL is given here:

\begin{tabular}{ l |l| l}
\hline\hline
Operators         & Notes                            & Associativity \\ \hline
$ . $             & Method call                      & left to right \\
$-$  $!$          & Unary minus and logical negation & right to left \\
$*$ $/$ $\%$      & Standard C meanings              & left to right \\
$+$ $-$           & Addition/subtraction             & left to right \\
$<$ $<=$ $>$ $>=$ &                                  &               \\
$!=$ $==$         &                                  &               \\
$\&\&$            &                                  &               \\
$||$              &                                  &
\end{tabular}

The subsections that follow use the model of the C Language Reference Manual to indicate the various types of expression.  As in that example. the highest-precedence forms of expression are listed first.  Since much of the material below is extremely straightforward, plain-English descriptions are supplemented by grammatical descriptions only when necessary.

\paragraph{Primary Expressions}

A primary expression consists of a constant (integer or string), an identifier, or a parenthesized expression.

\paragraph{Function, Method, and Mapper calls}

Arguments to functions, methods and mappers are evaluated in applicative order, left to right within a given list.  (Arguments are also passed by value, not by reference.)  Depending on the function or method in question, functions and methods may return values of any type, including boolean values (which cannot be assigned to variables); mappers, by their nature, always return patterns.

\emph{arglist}: \textbf ( \emph{expression} \textbf ) $|$  \textbf ( \emph{expression}
\textbf{,} \emph{arglist} \textbf )

\emph{method-call} : \emph{identifier} \textbf . \emph{identifier arglist}

\emph{function-call} : \emph{identifier} \emph{arglist}

\emph{mapper-call}:  \textbf{map} \emph{arglist mapper}

\emph{mapper} : \emph{identifier} $|$ \emph{block}

\emph{block} : \textbf{\{} \emph{statement-list} \textbf{\}}

\emph{statement-list}: \emph{statement} $|$ \emph{statement statement-list}

\paragraph{Unary operations}

The unary operations in DruL are arithmetic and logical negation (unary $-$ and $!$).  Since DruL is strictly typed, arithmetic negation can only be applied to integer values, and logical negation to boolean values.

\paragraph{Standard arithmetic operations}

Expressions may use the standard binary arithmetic operators ($+$, $-$, $*$ and $/$), with their standard precedence.  It is required that both of the operands in such an expression be integer values.

\paragraph{Comparison operations}

As in most C-family languages (and as shown in the precedence table above), relational operators have precedence over equality tests.  These operators return boolean values, which can be used in \textbf{if} statements but cannot be assigned to variables.

Relational tests may be used on integer values only; equality tests can be used on variables of any type, but in the case of patterns and clips, they will only report whether the two variables being tested are  aliases of the same object, not any deeper notion of equivalence.

\paragraph{Logical combination operations}

Here again we follow the conventions of C, and give $\&\&$ precedence over $||$.  These operators require their operands to be boolean values, and return boolean values.

\subsubsection{Assignment Statements}

Assignment in DruL is not a simple operator to be placed in the middle of an expression.  Rather, it is a separate type of statement, which may appear anywhere another statement may appear.

\begin{center}
\emph{assignment-statement}: \emph{identifier} \textbf{=}  \emph{expression-statement}
\end{center}

Assignment is polymorphic: the same syntax is used to assign variables to
integers, patterns and clips. Furthermore, due to DruL's dynamic typing,
a variable may be reassigned to a different type.

\subsubsection{Selection Statements}

Selection statements in DruL take the following form: the string \textbf{if}, followed by an expression that returns a boolean result, enclosed in parentheses, followed by an open-brace (``\{''), one or more statements, and a close-brace (``\}'').  This may optionally be followed by one or more \textbf{elseif}s, which are also followed by parentheses and a block of statements, and one (optional) \textbf{else}, which omits the test expression but is also followed by a block of statements.

\emph{selection-statement} : \textbf{if ( } \emph{boolean-expression} \textbf{ )}\emph{block if-tail}

\emph{if-tail} : $\epsilon$ $|$ \emph{if-middle} $|$
\emph{if-middle} \textbf{else \{} \emph{statement-list} \textbf{\}}

\emph{if-middle} : $\epsilon$ $|$ \textbf{elseif ( } \emph{boolean-expression} \textbf{ ) \{} \emph{statement-list} \textbf{\}} \emph{if-middle}

\subsubsection{Mapper Definition Statements}\label{MapperDefinitionStatement}

A mapper definition consists of the word \textbf{Mapper}, followed by an identifier, followed by a parenthesized list of comma-separated identifiers, followed by a block.

\emph{mapper-definition} : \textbf{mapper} \emph{identifier namelist block}

\emph{namelist}: \textbf ( \emph{identifier} \textbf ) $|$  \textbf ( \emph{identifier}
\textbf{,} \emph{namelist} \textbf )

\subsubsection{Return statements}

A return statement can only appear inside the statement block of a named or anonymous mapper:

\emph{return} : \textbf{return} \emph{expression}

If this statement is reached, the value of \emph{expression} will be the output of this iteration of the mapper block.  Accordingly, this expression must evaluate to either a pattern or a beat value.

\subsubsection{Instrument definition}

This is a special statement that closely resembles a function call:

\emph{instrument-definition}: \textbf{instruments (} \emph{arglist} \textbf{)}

The arguments to this pseudo-function must all evaluate to strings.  See section \ref{instrSection} for
a detailed discussion.

\subsection{Blocks, namespace and scoping}

\subsubsection{Blocks}

DruL has a limited block structure: only in the context of an \textbf{if}/\textbf{elseif}/\textbf{else} sequence or a Mapper Definition statement is a new block needed or allowed.  In these cases, curly braces (``\{\}'') are used to delimit the statement-sequence that falls within the block, and they must contain one or more statements.

Mapper definitions define a new closed scope (one from within which externally defined variables are not visible); \textbf{if} blocks do not define a new scope, so all variables used within them are visible to the enclosing  block, and vise-versa.

\subsubsection{Namespace}

DruL has one namespace shared by variables, built-in functions and
mapper names. Additionally, each type has an associated namespace for methods.  Technically speaking, mappers are values like any other, and their names can be re-used, but this is strongly discouraged.

\subsubsection{Scoping}

Variables in DruL are dynamically scoped.
DruL has one top level scope, and one scope for each mapper that the program enters
(named or anonymous). Mappers may call each other (or themselves) recursively, and may be defined within other mappers, so a hierarchy of substantial depth can (in principle) be achieved.
Within each scope, a program has read-only access to all variables and mapper names
defined in the scopes above it in this hierarchy: attempts to assign to a variable from an outer scope
will produce a new variable in the inner scope, which masks the original variable.

\subsection{Patterns and pattern operations}

\subsubsection{Patterns}

A pattern is a sequence of beats.  Each beat can be a note or a rest.
To define a pattern, DruL uses `0' for rests and `1' for notes.
A pattern can be created in the following way:
\begin{verbatim}
  p1 = pattern("101010");
\end{verbatim}
which represents the sequence {note, rest, note, rest, note, rest}.
Its length is $6$.

There are built-in functions and methods on patterns included in DruL.

Patterns can be concatenated to form new patterns.  The \textbf{concat} function can take any positive number of pattern arguments.  Patterns are concatenated from left to right.
\begin{verbatim}
  pcat = concat(p1 pattern("111000") pattern("1"));
\end{verbatim}
\textit{pcat} will be equal to 1010101110001.


The \textbf{repeat} method is a shortcut to concatenate the same pattern many times:
\begin{verbatim}
  pcat2 = concat(p1, p1, p1);
  pcat3 = pattern("101010").repeat(3);
  pcat4 = p1.repeat(3);
\end{verbatim}
Note that \textit{pcat2}, \textit{pcat3}, and \textit{pcat4} are all equivalent.

The \textbf{length} method gives the length of a pattern.
\begin{verbatim}
  len = p1.length();
\end{verbatim}
The value of \textit{len} is $6$.

The \textbf{slice} method gives you a subpattern from a pattern.
It takes two arguments: first is index (starting at $1$) and second is
length of the desired subpattern. Requesting a subpattern out of range will
raise an error.  Example:
\begin{verbatim}
  psub = pattern("101010").slice(2, 3);
\end{verbatim}
\textit{psub} is ``010''.

The \textbf{reverse} method returns the reverse of a pattern.
It doesn't take any arguments.
\begin{verbatim}
  preverse = pattern("111000").reverse();
\end{verbatim}
\textit{preverse} is ``000111''.

Finally, you can have an empty pattern of length $0$:
\begin{verbatim}
  p8 = pattern("");
\end{verbatim}

\subsubsection{Map}\label{MapSection}

The \textbf{map} construct is used to create new patterns from existing ones.
\textbf{map} performs an operation iteratively on a set of patterns.
The beats in the patterns are iterated over from left to right.
The output of a map is a new pattern.  For example:
\begin{verbatim}
p9  = pattern("101");
p10 = map(p9)
{
    if ($1.note()) { return pattern("11"); }
    else           { return pattern("0");  }
};
\end{verbatim}
p10 is ``11011''.

\textbf{map} takes a sequence of pattern arguments and followed by a mapper function.  In the above example the mapper function is defined anonymously within curly braces.

Within a mapper function, the current beat of each pattern argument is aliased to the special mapper variables \$1, \$2, \$3... and so on.  This notation is mandatory in anonymous mapper functions such as the example above. If you use \$N while there is fewer than N arguments, DruL will raise an error.

DruL uses the \textbf{beat} methods \textbf{note}, \textbf{rest} and \textbf{null} to check whether
the current beat is a note, a rest, or null.
\textit{\$1.note} returns \textbf{true} if there is a note on the current beat in the first pattern argument, and \textbf{false} otherwise.

One can use the \textbf{beat} method \textbf{asPattern} to convert as beat to a pattern.  This way, one can then make use
of functions and methods of patterns.  For example:
\begin{verbatim}
p11 = map( pattern("1111") )
{
    return concat($1.asPattern(), pattern("0"));
};
\end{verbatim}
p11 is ``10101010''.

DruL uses the \textbf{beat} methods \textbf{prev} and \textbf{next} to access the previous and following beats of the pattern to which a given beat belongs.  These methods can be passed a single argument which specifies how far forward or back in the pattern to go.  For example:
\begin{verbatim}
p12 = map( pattern("1101") )
{
    if ($1.note() && $1.next(1).note()) { return pattern("1"); }
    else                                { return pattern("0"); }
};
\end{verbatim}
p12 is ``1000''.

\textbf{next} may return a NULL beat as it does when called in the last iteration of the above example.  When used with a NULL beat, both the \textbf{note} and \textbf{rest} methods will return \textit{false}.

If you call map on multiple patterns that are not of the same length, the shorter patterns will be padded with NULL beats.

By default, an empty pattern is returned for each iteration.

Each new pattern constructed by map begins as an empty string.  As the pattern arugments are iterated over, the return values of the mapper function (which are also patterns) are concatenated onto the end of the new pattern.

Variables defined in a mapper function are garbage collected at the end of the map.

\subsubsection{Mapper}

Mapper functions may also be defined with a name, to be used elsewhere in the program.

For example, the above example could have been written in the following way:
\begin{verbatim}
mapper myMapper(p)
{
    if (p.note() && p.next(1).note()) { return pattern("1"); }
    else                              { return pattern("0"); }
};
p12 = map(pattern("1101")) myMapper;
\end{verbatim}

Recall from section \ref{MapperDefinitionStatement} that a  Mapper definition includes a name for the mapper and  a \emph{namelist} of formal arguments.
When a named mapper is used in a \textbf{map} call, each pattern that is passed to the \textbf{map} is associated with the corresponding name in the \emph{namelist} in the mapper's definition.
Then, within the body of the mapper, the current beat of each pattern is aliased to that name, as well as to ``\$$n$''.

A mapper function must be defined before it is used.

\subsection{Clips}

\subsubsection{Instruments}\label{instrSection}

Before we define any clips, we must tell the compiler what instruments they will use.
This can only be done once per program, and uses the instruments function.
(Technically speaking, this is not a function but a special statement type that uses a function-like syntax.  The distinction is largely academic, however.)
This function can take a variable number of arguments.  Each argument is the name of an instrument to be defined.  In the example below, four instruments are defined:

\begin{verbatim}
instruments(``hihat'',``bassdrum'',``crash'',``snare'');
\end{verbatim}

Instruments must be defined before any clips have been defined. This function
can only be called once. Also, it cannot be called from inside a mapper.

\subsubsection{Clips}

A clip represents a collection of patterns to be played in parallel, where each pattern is played on a single instrument.

Once the instruments are defined, we can create a clip from our existing patterns, using an
associative-array notation:

\begin{verbatim}
clip1 = clip
(
    "bassdrum" <- downbeats,
    "hihat"    <- alternate_beats
);
\end{verbatim}
The same result can be achieved by simply listing the patterns for each instrument in the order they are defined using the \textbf{instruments} function:
\begin{verbatim}
clip2 = clip
(
    alternate_beats,
    downbeats
    // remaining instruments have an empty beat-pattern
);

\end{verbatim}
The patterns passed into clips are passed by value, not by reference.

Clips also have a small selection of output methods, discussed in the section below.

\subsection{Outputs}

DruL has two kinds of outputs: any data structure can be printed to standard output for debugging purposes, and clips may be output into files as text or using some
more complex representation, such as MIDI or Lilypond (for PDF conversion).

\subsubsection{Standard output}

The \textbf{print} statement displays any type to the standard
output, including strings. For example:
\begin{verbatim}
  print ("DruL");
  print (pattern("01"));
\end{verbatim}

The representation of a string is the string itself. The representation
of a pattern is the string that would have been used to initialize
the pattern. For example, if we have a pattern
\begin{verbatim}
  p = pattern("01").repeat(2);
  print(p);
\end{verbatim}
The output is "0101";

The \textbf{print} function always include a platform-appropriate line
ending.

\subsubsection{Text file}

Using the same format as is used by \textbf{print}, DruL can print a text representation of a clip to a file
using the \textbf{outputText} method of the clip:

\begin{verbatim}
myClip.outputText("myfile.txt");
\end{verbatim}

The file being written to is truncated if it exists, and created if it does not exist.

\subsubsection{MIDI file}

The method \textbf{outputMidi} works similarly, but in addition to the filename, it requires a tempo for the MIDI file to be produced (in beats per minute--this must be a positive integer).

\begin{verbatim}
myClip.outputMidi(``myfile.mid'',120);
\end{verbatim}

The transformation from clip to MIDI may rely
on external libraries like MIDGE\footnote{\url{http://www.undef.org.uk/code/midge/}}. There is no guarantee on which of the three existing MIDI formats is used.
DruL tries to match its instrument definition with MIDI instruments definitions
using the names. If no match can be found, DruL will use a default MIDI instrument (first one is cow bell).

\subsubsection{Lilypond file}

The clip method \textbf{outputLilypond} operates similarly to the above, but takes a title (to be printed at the top of the page of typeset music) as an optional second argument:

\begin{verbatim}
myClip.outputLilypond(``myfile.ly'', ``My New Drum Loop'');
\end{verbatim}

For best results, the resulting Lilypond file will need to be typeset using an external program (Lilypond, one presumes).

\clearpage
