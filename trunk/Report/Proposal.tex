
\chapter{Introduction}

DruL stands for ``Drumming Language''.  It is a programming language designed for composing drum music.  It is common these days for drum beat composers to create drum parts using computer software (e.g. FL Studio).  Creating drums parts with these programs often involes of lot of tedious ``pointing and clicking'' (especially when making longer drum parts).  DruL was designed to give the composer the ability to automate much of this tedium.  There already exist other more general-purpose music programming languages (e.g. ChucK, SuperCollider, Nyquist, Haskore).  These languages are complicated by note pitches, durations, and audio effects.  DruL is unconcerned with these things and focuses soley on allowing the drum composer to define and manipulate beat patterns.

DruL meets the needs of an algorithmic drum-composer with the beat, pattern, and clip domain-specific data-types.  A beat is either a note or a rest.  A pattern is a sequence of beats.  A clip is a mapping of patterns to instruments.  A composer starts by defining some starting patterns.  These patterns can then be individually processed by built-in DruL functions to produce new patterns.  Alternately, the composer may define and use new functions called mappers.  Composers then apply their mappers to patterns, iterating over the beats of one or more patterns at a time, building up a new pattern along the way.  Once the composer has a set of patterns with which they are happy, they can define their desired set of instruments (e.g. hi-hat, snare, bass drum, cowbell, etc.).  With the instruments defined, the composer can use the clip constuct to assign a pattern to each instrument.  Finally, a clip can be output to a MIDI file, playable by many multimedia players.
