\documentclass[11pt,twoside]{article}
 
\usepackage{hyperref}
\usepackage{graphicx}
\usepackage{amsmath}

\pagestyle{plain}

\setlength{\voffset}{0pt}
\setlength{\marginparwidth}{0pt}
\setlength{\oddsidemargin}{30pt}
\setlength{\evensidemargin}{30pt}
\setlength{\textwidth}{400pt}
\setlength{\parindent}{0pt}
\setlength{\parskip}{10pt plus 2pt minus 1pt}

\title{\textbf{DruL} Reference Manual\\
\vspace{1cm}
COMS W4115: Programming Language and Translators}
\author{Team Leader: Rob Stewart (rs2660) \and Thierry Bertin-Mahieux (tb2332) \and Benjamin Warfield (bbw2108) \and Waseem Ilahi (wki2001)}

\begin{document}
\maketitle
\begin{center}
\end{center}

\vspace{3cm}

\begin{figure}[h]
\begin{center}
\includegraphics[width=.2\columnwidth]{Water_Drop.pdf}
\end{center}
\end{figure}


\newpage


\section{Introduction}
DruL is a programming language designed for composing drum music.  Unlike other more general-purpose music programming languages (ChucK, SuperCollider, Nyquist, Haskore), DruL's focus is on defining and manipulating beat patterns and is unconcerned with pitches, sound durations, or audio effects.  DruL is mainly an imperative programming language, however it borrows ideas (map and filter) from the functional paradigm.  In additions to integers, DruL's main datatypes are pattern and clip. Instruments are defined as constants.

A pattern is essentially an object that holds binary, discrete, time-series data.  At each discrete-time step, which will henceforth refer to as a beat, there is either a note or a rest.  For the non-musically inclined, a note represents sound produced by the striking of a drum (or similar instrument) and a rest represents the absence of any such sound.  Patterns are immutable.  When a pattern is manipulated, the target pattern remains intact and a new copy is created.

An instrument is one of a pre-defined set of sounds (e.g. drum notes) that can occupy a single beat.

A clip is a mapping of patterns to instruments. Clips are processed in sequence as the program runs to produce output which may be audio, sheet-music notation, or a MIDI file.

DruL is a strictly and staticly typed language. However, types are not explicitly declared, they are inferred.

DruL programs do not contain any loops or user-defined functions.  All pattern 
and clip creation and manipulation is done using the map construct described below.

\section{Language Manual}

\subsection{Lexical Conventions}

\subsubsection{Comments}
Comments start with // and everything after them on that line
is considered a comment and is ignored. New line marks the end
of a comment.
\subsubsection{Whitespace}
Spaces, tab, end of line, and return are all considered the same 
and their only used is to seperate the tokens.

\subsubsection{Characters}
DruL uses the ASCII character set.

\subsubsection{Identifiers}
An identifier may be defined using any uppercase or lowercase character
or an underscore, followed by any set of uppercase or lowercase character,
underscore, and digits ($0$ through $9$). The maximum length of an identifier
is $64$.

\subsubsection{Keywords}


\begin{table}[htb]
\begin{center} 
\begin{tabular}{llll} 
  NULL &    rand &    clip &        mapper \\
  if &      pattern & instrument &  print \\
  elseif &  concat &  length &      output \\
  else &    slice &   map &         return
\end{tabular}
%\caption{\small{.}}
\label{tab:keywords}
\end{center} 
\end{table}


\subsection{Types}

There are $3$ types in DruL: \textbf{integer}s, \textbf{pattern}s,
and \textbf{clip}s.  In addition, string constants may be used in DruL source code, but there is no variable type to which they can be directly assigned.
Values in DruL are strongly typed, but the type of a variable is determined dynamically.

\subsubsection{integer}
All integers are base 10, and may optionally be preceded by a sign ({\tt \textbf + -}).
Any sequence of digits 
($0$ through $9$) is valid.  Leading $0$s are ignored, so a sequence such as $0000123$  is interpreted as $123$.

\subsubsection{pattern}
A pattern is a series of events and silences.
It can be of any non-negative length.


\subsubsection{clip}
A clip is a set of patterns associated with instruments.

\subsubsection{string}

A string constant begins with an ASCII double-quote character, continues with an arbitrary sequence of ASCII characters other than $\\$ and $''$, and concludes with another $''$ character.  If a $\\$ or $''$ character is desired, it can be escaped using the $\\$ character.

\subsection{Statements}

\subsubsection{Expressions}


\subsubsection{Blocks}
The curly braces mark the beginning and end of a block.

\subsubsection{Logical operators}
$\&\&$, $||$, $==$, etc


\subsubsection{Conditional}


\subsubsection{Assignments}
We can assign any type to any identifier using the equal ($=$) sign.
???

\subsection{Patterns and pattern operations}

\subsubsection{Patterns}

\subsubsection{Map}

\subsubsection{Mapper}
Mapper has to be declared before it is used in the file.

\subsection{Clips}

\subsubsection{Instruments}


\subsubsection{Clips}

\subsection{Outputs}

\subsubsection{Standard output}

\subsubsection{Text}

\subsubsection{MIDI}

\section{Examples}
In this section we give examples of what DruL code will look like, in the form
of a tutorial.

\subsection{Integers}
Integers are part of our language. Unlike patterns and clips, they are mutable.
\begin{verbatim}
a = 3;
b = a + 2;
c = b * 12;
\end{verbatim}

\subsection{Pattern}
Patterns are the data type the programmer will likely spend most of their time dealing with.  For convenience, the programmer can supply a string constant made up of 1s and 0s, which will be translated into a pattern: if the character is a 1, there is a note on the corresponding beat; if 0, a rest.
\begin{verbatim}
p1 = pattern("101010");
\end{verbatim}
Patterns can be concatenated to form new patterns:
\begin{verbatim}
pcat = concat(p1 pattern("111000") pattern("1"));
\end{verbatim}
\textit{pcat} will be equal to 1010101110001.

There is also a shortcut to concatenate the same pattern many times:
\begin{verbatim}
pcat2 = concat(p1 p1 p1);
pcat3 = pattern("101010").repeat(3);
pcat4 = p1.repeat(3);
\end{verbatim}
\textit{pcat2}, \textit{pcat3}, and \textit{pcat4} are all equivalent.

\subsection{Map}
Of course, we will not hardcode every pattern we want to create. We use
map to create meaningful new patterns from existing ones:
\begin{verbatim}
p2 = map (p1)
{
    if (p1.note) { pattern("11"); }
    else         { pattern("0");  }
};
\end{verbatim}
This will create the following pattern: 110110110. The goal of a map
is to easily iterate over a pattern. \textit{p1.note} returns
\textit{true} if there is a note on the current beat, \textit{false} otherwise.
If you call map on multiple patterns that are not of the same length,
the shorter patterns will be padded with \textit{NULL} beats.\\

\subsection{Mapper}
For ease of use, you can define a \textit{mapper} that contains the behaviour
used by \textit{map}. We create \textit{p3}, which is the same as
\textit{p2}:
\begin{verbatim}
mapper myMapper (p1)
{
    if (p1.note) { return pattern("11"); }
    else         { return pattern("0");  }
}

p3 = map (p1) myMapper;
\end{verbatim}
\textit{mapper} will be very important when building a standard library
for the language.

\subsection{More complex examples}
Now that we have a proper syntax, let's get to more complicated examples.
We introduce $2$ new features that can be used inside a \textit{map}:
\textit{prev} and \textit{next}. They give you access to earlier
and later beats in a pattern, using the syntax \textit{p.prev(n)} and
\textit{p.next(n)}. 
%Also, for a pattern \textit{p}, \textit{p.rest} is
%true if and only if we did not reach the end of this pattern.

\textbf{reduction}: accelerate by cutting one beat out of two
\begin{verbatim}
downbeats = pattern("1000100010001000");
alternate_beats = pattern("10").repeat(8);
downbeat_diminution = map(downbeats alternate_beats)
{
    if     (alternate_beats.rest) { return pattern("");  } // pattern of length 0
    elseif (downbeats.note)       { return pattern("1"); }
    else                          { return pattern("0"); }
}
\end{verbatim}
output is: 10101010.

\textbf{improved reduction}: putting a rest (0) only if the $2$ original beats were rest
\begin{verbatim}
// this will map "1001100110011001" to "11111111", rather than "10101010"
one_and_four = pattern("1001100110011001");
alternate_beats = pattern("10").repeat(8);
improved_diminution = map(one_and_four alternate_beats)
{
    if     (alternate_beats.rest)      { return pattern("");  } // still required
    elseif (one_and_four.note)         { return pattern("1"); }
    elseif (one_and_four.next(1).note) { return pattern("1"); }
    else                               { return pattern("0"); }
};
\end{verbatim}

\subsection{Instruments and Clips}

Now that we have a large and varied collection of patterns, we can show how to combine those patterns into clips.  

Before we define any clips, we must tell the compiler what instruments they will use.  This can only be done once per program, and uses the \textit{instruments} function:

\begin{verbatim}
instruments(hihat bassdrum crash snare);
\end{verbatim}

Once the instruments are defined, we can create a clip from our existing patterns, using an
associative-array notation:

\begin{verbatim}
clip1 = clip
(
    bassdrum = downbeats
    hihat    = alternate_beats
);
\end{verbatim}

The same result can be achieved by simply listing the patterns for each instrument in the order they are defined in the \textit{instruments} declaration:
\begin{verbatim}
clip2 = clip
(
    alternate_beats
    downbeats
    // remaining instruments have an empty beat-pattern
);

\end{verbatim}



\end{document}
